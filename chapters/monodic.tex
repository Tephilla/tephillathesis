\chapter{The monodic logic}
\section{Preliminaries}

The set of places is composed of the two disjoint sets of server places $P_s$ and client places $P_c$, i.e $P_s \cap P_c = \emptyset$ and $P= P_s \cup P_c$. 
Let $L$ be the labeling function mapping the places to the properties satisfied in them. As a slight abuse of notation, we use $L(p)$ and $p$ interchangeably for some $p\in P$. With respect to the case study, we may enumerate the names of the places. However, for simplicity, they are abstracted out into place indices using a labeling function.
Clients are mapped to some client state $cs_i$ which changes dynamically over time, for a client with identifier $i$.
We thereby have a tuple (client vector) containing \textit{bound} number of clients, $CS_\lambda=\langle cs_0, cs_1, \cdots, cs_{bound}\rangle$ where $cs_i \rightarrow P_c \cup \{-2,-1\}$ and the client identifier $i \in \{\mathbb{N}\setminus \{0\}\}$ and $\lambda$ is the parameter denoting time instance and in this case, the execution steps of the net. For instance, consider $CS_3=\langle -2, 1, -1\rangle$ which denotes that there are three client processes  - the client that exited the system, the client that is in the client place $1$, and a client that has not yet entered the system, where $\lambda=3$. 

Similarly, we have a server variable describing the server state with respect to the execution steps. Since this is a single server system there is exactly one variable $SS_\lambda$ that varies with respect to the execution steps $\lambda$ as previously shown. If the model were to have multiple servers, this would be a vector. 

\textbf{Firing equation}
Given $r$ clients in the system and $\vert P\vert=e$, the marking of the net at time instance $\lambda$ is given as \[M_\lambda=\langle s_{p_0}, s_{p_1}, \ldots, s_{p_e}\rangle_\lambda\] where $s_{p_i}$ denotes the set of clients identifiers in place $p_i$.
The formula encoding the marking $M_\lambda$ is a conjunction of the place variables and is denoted by $[M_\lambda]$.
\todo{what is AND of sets?}

\[[M_\lambda]= \underset{0 \le i \le e}{\bigwedge} s_{p_i}\]

Similarly, the encoded formula of the  configuration $M_\lambda^{'}$ is given as $[M_\lambda^{'}]$.

The transition function of the net is a disjunction of the following subformulas:
\begin{enumerate}
	\item formula denoting source transition firing $\phi_{src}$
	\item formula denoting server transition firing $\phi_{s}$
	\item formula denoting client transition firing $\phi_{c}$
	\item formula denoting sink transition firing $\phi_{sink}$
	%\item post condition where there is exactly one change in the marking after firing of a transition $\phi_{post}$
\end{enumerate}
Therefore, the formula for the firing equation is as follows:
\[\phi_{firing} = \phi_{src} \lor \phi_{s} \lor \phi_{c} \lor \phi_{sink}\]
%In the net, the source transition has outgoing arcs labeled by $\nu$. 
\dfn{Source transition firing rule: }
The formula denoting the source transition firing is given by $\phi_{src}$. Clients are dynamically generated by source transitions, where the outgoing arc is labeled $\nu$. Let $ctr$ be the global counter for the clients in the system, which changes dynamically. Let $out_{src}$ be the set of output places of the source transition.
Let $\{p_0, p_1, p_i, p_j, \ldots, p_k\}$ be the places in the net. Let $M_0=\langle \emptyset,\emptyset, \ldots, \emptyset\rangle$ be the initial marking (at $\lambda=0$), and the counter is initialized $ctr=0$. Assume that $out_{src}=\{p_i, p_j, \ldots, p_k\}$ are the output places of the source transition and each of the sets are initially empty $\{s_{p_i}, s_{p_j}, \ldots, s_{p_k}\}=\emptyset$.
 On firing the source transition exactly once, for each place in $out_{src}$, the resultant set is the original set unioned with the newly generated client identifier $ctr^{'}$ i.e,
$\{s_{p_i}^{'}, s_{p_j}^{'}, \ldots, s_{p_k}^{'}\}=\{\{s_{p_i}\cup ctr^{'}\}, \{s_{p_j}\cup ctr^{'}\}, \ldots, \{s_{p_k}\cup ctr^{'}\}\}$ where $ctr^{'}=ctr+1$.
  Hence, the new configuration $M_\lambda^{'}$ is given as:
\begin{align*} 
M_\lambda^{'}= M_{\lambda +1}= M_1 &= \langle \emptyset, \emptyset, s_{p_i}^{'}, s_{p_j}^{'}, \ldots, s_{p_k}^{'} \rangle \\
&= \langle \emptyset, \emptyset, \{s_{p_i}\cup ctr^{'}\}, \{s_{p_j}\cup ctr^{'}\}, \ldots,\{s_{p_k}\cup ctr^{'}\}\rangle \\
&= \langle \emptyset, \emptyset, 1, 1, \ldots, 1\rangle
\end{align*} 

We refer to the resultant marking as $M_\lambda^{'}$ instead of $M_{\lambda +1}$ for readability. 
\noindent In general, for each ${p_i}\in out_{src}$
\[M_\lambda^{'}=\langle s_{p_0}, s_{p_1}, \{s_{p_i}\cup ctr^{'}\}, \{s_{p_j}\cup ctr^{'}\}, \ldots, \{s_{p_k}\cup ctr^{'}\}, \ldots, s_{p_e}\rangle_{\lambda}\]

The formula encoding $M_\lambda$ and $M_\lambda^{'}$ are given as $[M_\lambda]$ and $[M_\lambda^{'}]$ respectively.

The formula $\phi_{src}=[M_\lambda] \wedge [M_\lambda^{'}]$. 

\dfn{Server transition firing rule: }
We call as server transitions, those transitions that have outgoing arcs labeled ``s".
Let $M_\lambda$ be the current configuration of the net at $\lambda$. Let $in_{s}=\{p_i, p_j, \ldots, p_k\}$ and $out_{s}=\{p_l, p_m, \ldots, p_n\}$ be the input and output places of a server transition respectively. \[M_\lambda=\langle s_{p_0}, s_{p_1}, s_{p_i}, s_{p_j}, \ldots, s_{p_k},  s_{p_l},  s_{p_m}, \ldots,  s_{p_n},s_{p_e}\rangle_{\lambda}\]



\noindent On firing the source transition exactly once, for each place $p_i \in in_{s}$, the resultant set contains one less agent, which is moved, say, $c_{m}$ s.t $\{\exists s_{p_i} \vert c_{m} \in s_{p_i}, p_i \in in_{s}\}$ and for each place $p_l \in out_{s}$, the resultant set contains the moved agent $c_{m}$ s.t $\{\exists s_{p_l} \vert c_{m} \in s_{p_l}, p_l \in out_{s}\}$. 


\noindent In general, the resultant configuration $M_\lambda^{'}$ is given by:
\[M_\lambda^{'}=\langle s_{p_0}, s_{p_1}, \{s_{p_i}\\\setminus c_{m}\}, \{s_{p_j}\setminus c_{m}\}, \ldots, \{s_{p_k}\setminus c_{m}\}, \ldots, \{s_{p_l}\cup c_{m}\},  \{s_{p_m}\cup c_{m}\}, \ldots,  \{s_{p_n}\cup c_{m}\},s_{p_e}\rangle_{\lambda}\]


The formula encoding $M_\lambda$ and $M_\lambda^{'}$ are given as $[M_\lambda]$ and $[M_\lambda^{'}]$ respectively.

The formula $\phi_{s}=[M_\lambda] \wedge [M_\lambda^{'}]$. 


\dfn{Client transition firing rule: }
We call as client transitions, those transitions that have outgoing arcs labeled ``c". Let $M_\lambda$ be the current configuration of the net at $\lambda$. Let $in_{c}=\{p_i, p_j, \ldots, p_k\}$ and $out_{c}=\{p_l, p_m, \ldots, p_n\}$ be the input and output places of a client transition respectively. \[M_\lambda=\langle s_{p_0}, s_{p_1}, s_{p_i}, s_{p_j}, \ldots, s_{p_k},s_{p_l},  s_{p_m}, \ldots,  s_{p_n},s_{p_e}\rangle_{\lambda}\]



\noindent On firing the source transition exactly once, for each place $p_i \in in_{c}$, the resultant set contains one less agent, which is moved, say, $c_{m}$ s.t $\{\exists s_{p_i} \vert c_{m} \in s_{p_i}, p_i \in in_{c}\}$ and for each place $p_l$ in $out_{c}$, the resultant set contains the moved agent $c_{m}$ s.t $\{\exists s_{p_l} \vert c_{m} \in s_{p_l}, p_l \in out_{c}\}$. 


\noindent In general, the resultant configuration $M_\lambda^{'}$ is given by:
\[M_\lambda^{'}=\langle s_{p_0}, s_{p_1}, \{s_{p_i}\\\setminus c_{m}\}, \{s_{p_j}\setminus c_{m}\}, \ldots, \{s_{p_k}\setminus c_{m}\}, \ldots, \{s_{p_l}\cup c_{m}\},  \{s_{p_m}\cup c_{m}\}, \ldots,  \{s_{p_n}\cup c_{m}\},s_{p_e}\rangle_{\lambda}\]



The formula encoding $M_\lambda$ and $M_\lambda^{'}$ are given as $[M_\lambda]$ and $[M_\lambda^{'}]$ respectively.

The formula $\phi_{c}=[M_\lambda] \wedge [M_\lambda^{'}]$. 

\dfn{Sink transition firing: }
The formula for firing the sink transition is described as follows. Assume that $in_{sink}=\{p_i, p_j, \ldots, p_k\}$ are the input places of the sink transition. 
On firing the sink transition exactly once, for each place $p_i \in in_{sink}$, the resultant set contains one less client identifier, which is sunk, say, $c_{sunk}$ s.t $\{\exists s_{p_i} \vert c_{sunk} \in s_{p_i}, p_i \in in_{sink}\}$


For the set of places in $in_{sink}$, we have the following resultant set:
$\{s_{p_i}^{'}, s_{p_j}^{'}, \ldots, s_{p_k}^{'}\}=\{s_{p_i}\setminus c_{sunk}\}, \{s_{p_j}\setminus c_{sunk}\}, \ldots, \{s_{p_k}\setminus c_{sunk}\}$.

\noindent In general, for each ${p_i}\in in_{sink}$ is modified and the resultant configuration $M_\lambda^{'}$ is given by:
\[M_\lambda^{'}=\langle s_{p_0}, s_{p_1}, \{s_{p_i}\\\setminus c_{sunk}\}, \{s_{p_j}\setminus c_{sunk}\}, \ldots, \{s_{p_k}\setminus c_{sunk}\}, \ldots, s_{p_e}\rangle_{\lambda}\]

The formula encoding $M_\lambda$ and $M_\lambda^{'}$ are given as $[M_\lambda]$ and $[M_\lambda^{'}]$ respectively.

The formula $\phi_{sink}=[M_\lambda] \wedge [M_\lambda^{'}]$. 


